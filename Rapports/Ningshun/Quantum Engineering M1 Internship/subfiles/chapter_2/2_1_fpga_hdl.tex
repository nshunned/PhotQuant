\paragraph{Field Programmable Gate Arrays}
An FPGA is an integrated circuit that comes with configurable logic blocks and other features. FPGAs differ from traditional integrated circuits in that they can be programmed and reprogrammed by users, such that they allow the creation of custom digital circuits by connecting the configurable blocks as needed. Hence they are ``field-programmable," indicating that their abilities are adjustable and not hardwired by the manufacturer. Additionally, FPGAs excel at parallel processing and can even outperform central processing units (CPUs) for certain tasks.

\paragraph{Hardware Design Language}

HDLs (such as VHDL or Verilog) are used to describe the behavior of digital circuits, which can be thought of as programming languages specifically tailored for hardware design. HDL code is compiled into a bit pattern file (or bitfile) that specifies the connections inside the FPGA, defining its behaviours. When the FPGA is booted, the bitfile is loaded into the circuit, making the necessary connections and configuring its functionality.

For the experimental development of hardware programs, the Vivado design softare envirement is used. It includes tools for design entry, synthesis, place and route, and verification and simulation.