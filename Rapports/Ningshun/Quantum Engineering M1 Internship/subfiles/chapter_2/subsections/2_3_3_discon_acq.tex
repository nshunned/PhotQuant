Deep Memory Acquisition on the Red Pitaya is limited to a continuous mode, with the number of samples to be collected by the 125 MS/s ADC as the only defining parameter. Currently, there are no functions in the list of supported SCPI and API commands that can customize acquisitions beyond a simple continuous fashion. However, referring to Figure \ref{fig:ch2_acquisition} as an example, we see that experimental data acquisition is put in a loop with roughly 250 $\mu$s between each run. Within this  250 $\mu$s experimental pause, we would like DMA to pause as well, in order to avoid acquiring nonsensical data that will not be used. Therefore, it becomes crucial to understand how the DMA feature, specifically through its list of supported SCPI and API commands, configures the Red Pitaya to acquire data via the ADC. The goal here is to be able to configure the Red Pitaya with home-made SCPI and API commands that will allow customized, discontinuous data acquisition.

This experimental endeavor is still ongoing.