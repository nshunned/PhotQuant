Currently in Master 1 Quantum Engineering at École Normale Supérieure — Université Paris Sciences et Lettres, I carried out my second semester internship in the Quantum Photonics team of Collège de France. The team is directed by Alexei \textsc{Ourjoumtsev}, and is composed of CNRS researcher Sébastien \textsc{Garcia}, PhD students Valentin \textsc{Magro} and Antoine \textsc{Covolo}, and two other interns Mathieu \textsc{Girard} and Yi \textsc{Li}.

The central project of the research team is the ``quantum engineering of light with Rydberg interaction in an ensemble of cold atoms in a cavity." The experimental apparatus of this endeavour aims to create coherent interactions between optical photons and a cold cloud of Rubidium located inside the cavity. This setup allows the team to control a superposition state between a Rydberg excitation and a photon in a coherent fashion \cite{vaneecloo2022}.

The goal of this internship was to facilitate and contribute to the software migration process from  LabVIEW, a proprietary language used in the experiment for programming applications for digital and analog inputs/outputs (I/O), to open-source languages such as Python or C. A successful migration away from LabVIEW that eliminates its usage in totality has the potential to increase code development efficiency, make digital and analog I/O processes more adaptable, and drastically reduce the cost of experimental operations.